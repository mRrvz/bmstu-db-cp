\chapter{Аналитическая часть}

В данном разделе описана структура рабочей программы дисциплины. Представлен анализ способов хранения данных и систем управления базами данных, оптимальных для решения поставленной задачи. Описаны проблемы кэшированных данных и представлены методы их решения.

\section{Формализация задачи}

Каждая дисциплина, преподаваемая в высшем учебном заведении, имеет свою рабочую программу. В ней хранятся различная информация о дисциплине: стандарт, содержание, объем, результаты обучения, перечень литературы, методические указания и прочее. Зачастую, у пользователей нет никаких автоматизированных инструментов для анализа и редактирования таких программ.

Дисциплина имеет свой федеральный государственный образовательный стандарт: 3+, 3++ и другие \cite{standard}. Образовательный стандарт -- это совокупность обязательных требований к образованию определенного уровня и (или) к профессии, специальности. Каждый образовательный стандарт для каждого направления подготовки обучаемого имеет свою компетенцию. Компетенция -- некоторый свод информации, о том что должен знать, уметь и какими навыками должен обладать выпускник, успешно осовевший дисциплину. 

Каждому направлению подготовки, но одной и той же рабочей программы дисциплины сопоставлены различные компетенции (которые, в свою очередь, различны в каждом образовательном стандарте). Например, есть два студента, успешно осовевшие дисциплину <<Физика>>. Один из них обучается по направлению подготовки <<Экономика>>, а второй по направлению <<Теплофизика и теоретическая теплотехника>>. Очевидно, что второй студент должен владеть большими знаниями о данной дисциплине.

Коды компетенций для каждого образовательного стандарта отличаются. К сожалению, не редки случаи, когда при переходе на новую образовательную программу меняется только код и название компетенции -- содержание компетенции остается абсолютно точно таким же. В связи с этим, возникает возможность автоматизации перевода документов рабочей программы дисциплины на новый стандарт.

Кроме того, имея данные о всех рабочих программах дисциплинах, можно адаптировать какие-либо из них под выбранные направления подготовки. Например, рассчитать оптимальную нагрузку по данной дисциплине для студентов обучающихся на данном направлении.

Далее, в качестве примера, будем рассматривать рабочую программу дисциплины <<Информатика>>, соответствующую образовательному стандарту 3++, разработанную и преподаваемую в МГТУ им. Н. Э. Баумана \cite{bmstu}.

\section{Структура рабочей программы дисциплины}

Структура файлов рабочих программ дисциплин можно разниться от ВУЗа к ВУЗу, но, внутри одного ВУЗа, скорее всего, все программы имеют одну и ту же (или схожую) структуру. Рабочая программа дисциплины <<Информатика>> имеет следующие разделы:

\begin{enumerate}
	\item титульный лист;
	\item планируемые результаты обучения по дисциплине, соотнесенные с планируемыми результатами освоения образовательной программы;
	\item место дисциплины в структуре образовательной программы;
	\item объем дисциплины;
	\item содержание дисциплины;
	\item учебно-методическое обеспечение самостоятельно работы;
	\item фонд оценочных средств для проведения текущего контроля и промежуточной аттестации студентов;
	\item перечень основной и дополнительной литературы;
	\item методические указания;
	\item перечень информационных технологий;
	\item описание материально-технической базы.
\end{enumerate}

Разделы №2, №4, №5 представленны в виде совокупности текстовой информации и таблиц (рис. \ref{img:rpd_example_01}). Остальные разделы представлены в виде текстовой информации (рис. \ref{img:rpd_example_02}).

\begin{figure}[h!]
	\begin{center}
		\includegraphics[scale=0.5]{img/rpd_example_02.png}
	\end{center}
	\captionsetup{justification=centering}
	\caption{Изображение таблицы c информацией о объеме дисциплины.}
	\label{img:rpd_example_01}
\end{figure}

\begin{figure}[h!]
	\begin{center}
		\includegraphics[scale=0.5]{img/rpd_example_01.png}
	\end{center}
	\captionsetup{justification=centering}
	\caption{Изображение текстовой информации о месте дисциплине в структуре образовательной программы.}
	\label{img:rpd_example_02}
\end{figure}

Интерес представляют разделы №1 (титульный лист), №2 (результаты обучения), №4 (объем дисциплины), №5 (содержание дисциплины) и №8 (перечень литературы). 

Первый раздел содержит общую информацию о курсе: название, образовательный стандарт и прочее. 

Раздел №2 содержит коды и описания компетенций для каждого направления подготовки. С помощью информации, полученной в этом разделе, можно попробовать автоматизировать перенос файла рабочей программы дисциплины с одной образовательной программы на другую.

В разделах №4 и №5 хранится информация о нагрузке и структуре дисциплины -- эта информация может пригодится для анализа нагрузки на студентов по выбранному направлению подготовки и структуризации рассматриваемой рабочей программы дисциплины.

\section{Базы данных и системы управления базами данных}

В задаче разбора и хранения информации рабочей программы дисциплины важную роль имеет выбор модели хранения данных. Для персистентного хранения данных используются базы данных \cite{database}. Для управления этими базами данных используется системы управления данных -- СУБД \cite{subd}. Система управления базами данных -- это совокупность программных и лингвистических средств общего или специального назначения, обеспечивающих управление созданием и использованием баз данных.

\section{Хранение данных о рабочих программах дисциплины}

Система, разрабатываемая в рамках курсового проекта, предполагает собой приложение, которое является микросервисом \cite{microservice} одной большой системы -- системы управления обучения. 

Предполагается, что доступ к разрабатываемому приложению будем иметь лишь только <<ядро>> этой системы. При этом, только у одного типа пользователя системы есть доступ к данным, хранящимся в приложении -- преподавателю. Состояние гонки (англ. Race condition \cite{race-condition}) можно исключить - каждый преподаватель работает только с информацией из файлов, которые он самостоятельно загрузил в базу данных.

Для хранения данных о рабочей программы дисциплины необходимо использовать строго структурированную и типизированную базу данных, потому что вся информация, предоставленная в файлах программы имеет чётко выраженную структуру, которая не будет меняться от дисциплины к дисциплине.

\subsection{Классификация баз данных по способу хранения}

Базы данных, по способу хранения, делятся на две группы -- строковые и колоночные. Каждый из этих типов служит для выполнения для определенного рода задач.

\noindent\textbf{Строковые базы данных}\\

Строковыми базами даных называются такие базы данных, записи которых в памяти представляются построчно. Строковые баз данных используются в транзакционных системах (англ. OLTP \cite{OLTP}). Для таких систем характерно большое количество коротких транзакций с операциями вставки, обновления и удаления данных - \texttt{INSERT}, \texttt{UPDATE}, \texttt{DELETE}. 

Основной упор в системах OLTP делается на очень быструю обработку запросов, поддержание целостности данных в средах с множественным доступом и эффективность, которая измеряется количеством транзакций в секунду. 

Схемой, используемой для хранения транзакционных баз данных, является модель сущностей, которая включает в себя запросы, обращающиеся к отдельным записям. Так же, в OLTP-системах есть подробные и текущие данных.\\

\noindent\textbf{Колоночные базы данных}\\

Колоночными базами данных называются базы данных, записи которых в памяти представляются по столбцам. Колоночные базы данных используется в аналитических системах (англ. OLAP \cite{olap}). OLAP характеризуется низким объемом транзакций, а запросы часто сложны и включают в себя агрегацию. Время отклика для таких систем является мерой эффективности.

OLAP-системы широко используются методами интеллектуального анализа данных. В таких базах есть агрегированные, исторические данные, хранящиеся в многомерных схемах. 

\subsection{Выбор модели хранения данных для решения задачи}

Для решения задачи построчное хранение данных преобладает над колоночным хранением по нескольким причинам:

\begin{itemize}
	\item задача предполагает постоянное добавление и изменение данных;
	\item задача предполагает быструю отзывчивость на запросы пользователя;
	\item задача не предполагает выполнения аналитических запросов;
\end{itemize}

\subsection{Обзор СУБД с построчным хранение}

В данном подразделе буду рассмотрены популярные построчные СУБД, которые могут быть использованы для реализации хранения в разрабатываемом программном продукте.\\

\textbf{PostgreSQL}\\

PostgreSQL \cite{postgresql} -- это свободно распространяемая объектно-реляционная система управления базами данных, наиболее развитая из открытых СУБД в мире и являющаяся реальной альтернативой коммерческим базам данных \cite{postgresql-fact}.

PostgreSQL предоставляет транзакции со свойствами атомарности, согласованности, изоляции, долговечности (ACID \cite{acid}), автоматически обновляемые представления, материализованные представления, триггеры, внешние ключи и хранимые процедуры. Данная СУБД предназначена для обработки ряда рабочих нагрузок, от отдельных компьютеров до хранилищ данных или веб-сервисов с множеством одновременных пользователей. 

Рассматриваемая СУБД управляет параллелизмом с помощью технологии управления многоверсионным параллелизмом (англ. MVCC \cite{mvcc}). Эта технология дает каждой транзакции <<снимок>> текущего состояния базы данных, позволяя вносить изменения, не затрагивая другие транзакции. Это в значительной степени устраняет необходимость в блокировках чтения (англ. read lock \cite{r-lock}) и гарантирует, что база данных поддерживает принципы ACID. \\

\textbf{Oracle}\\

Oracle \cite{oracle} -- 

\textbf{MySQL}\\

MySQL \cite{mysql} --

\subsection{Выбор СУБД для решения задачи}

Для решения задачи была выбрана СУБД PostgreSQL, потому что данная СУБД имеет поддержку языка plpython3u \cite{plpython3u}, который упрощает процесс интеграции базы данных в разрабатываемое приложение. Кроме того, PostgreSQL проста в развертывании.

\section{Кэширование данных}

\subsection{Проблемы кэширования данных}

\subsection{Обзор in-memory СУБД}

\subsection{Выбор СУБД для решения задачи}

\section{Формализация данных}

\subsection{База данных рабочих программ дисциплин}

\subsection{База данных кэшируемой информации}

База данных кэшируемых значений должна хранить значения без дополнительной обработки. Данные должны быть актуальны и синхронизированны с основным хранилищем: кэш должен обновляться после каждой транзакции. Кроме того, нужно ограничить размер кэша и добавить вытеснение из него, например, с помощью политики вытеснения LRU \cite{lru} (Last Recently Used).

\section*{Вывод}

В данном разделе:

\begin{itemize}
 \item рассмотрена структура рабочей программы дисциплины и выявлены её наиболее интересные части;
 \item проанализированы способы хранения информации для система и выбраны оптимальные способы для решения поставленной задачи; 
 \item проведен анализ СУБД, используемых для решения задачи и также выбраны оптимальные информационные системы; 
 \item рассмотрена проблема актуальности кэшируемых данных и предложенно ее решение;
 \item формализованны данные, используемые в системе.
\end{itemize}

