\chapter{Исследовательская часть}

В данном разделе представлена постановка эксперимента по сравнению занимаемого времени для получения данных из хранилища с использованием и без использования кэширования.

\section{Постановка эксперимента}

В данном подразделе представлены цель, описание и результаты эксперимента.

\subsection{Цель эксперимента}

Целью эксперимента является сравнение времени, требуемого для получения данных о рабочей программе дисциплины с и без использованием кэширования данных.

\subsection{Описание эксперимента}

Сравнить занимаемое время можно при помощи отключения реализованного механизма кэширования. Для этого будет достаточно отключить базу данных, хранящую данные о кэшировании и каждый раз выполнять запрос напрямую к базе данных хранящую информацию о рабочих программах дисциплин.

Для проведения будут использоваться кэши разных размеров, а так же разное количество запрашиваемых рабочих програм дисциплин. Будут произведены операции, для того чтобы запрашиваемые РПД могли оказаться в кэше.

В поставленном эксперименте одна рабочая дисциплина состоит из 32 единиц (таблиц).

\subsection{Результат эксперимента}

В таблицах \ref{tbl:experiment1} - \ref{tbl:experiment3} представлены результаты поставленного эксперимента. 

\begin{table}[H]
	\centering
	\caption{Результаты сравнения времени, необходимого для получениях данных без кэширования и с кэшированием (размер кэша - 100 единиц)}
	\label{tbl:experiment1}
	\resizebox{\textwidth}{!}{%
		\begin{tabular}{|l|l|l|}
			\hline
			\textbf{\begin{tabular}[c]{@{}c@{}}Количество \\ РПД\end{tabular}} & \textbf{Время без кэширования, мс} & \textbf{Время с кэшированием, мс} \\ \hline
			1 & 62012 & 4554 \\ \hline
			5 & 366500 & 340250 \\ \hline
			10 & 751340 & 702250 \\ \hline
			25 & 2657210 & 2584172 \\ \hline
			100 & 9750742 & 9209781 \\ \hline
		\end{tabular}%
	}
\end{table}

\begin{table}[H]
	\centering
	\caption{Результаты сравнения времени, необходимого для получениях данных без кэширования и с кэшированием (размер кэша - 1000 единиц)}
	\label{tbl:experiment2}
	\resizebox{\textwidth}{!}{%
		\begin{tabular}{|l|l|l|}
			\hline
			\textbf{\begin{tabular}[c]{@{}c@{}}Количество \\ РПД\end{tabular}} & \textbf{Время без кэширования, мс} & \textbf{Время с кэшированием, мс} \\ \hline
			1 & 70233 & 4301 \\ \hline
			5 & 398213 & 28231 \\ \hline
			10 & 720304 & 50300  \\ \hline
			25 & 2011763 & 175680 \\ \hline
			100 & 9542401 & 7001307 \\ \hline
		\end{tabular}%
	}
\end{table}

\begin{table}[H]
	\centering
	\caption{Результаты сравнения времени, необходимого для получениях данных без кэширования и с кэшированием (размер кэша - 5000 единиц)}
	\label{tbl:experiment3}
	\resizebox{\textwidth}{!}{%
		\begin{tabular}{|l|l|l|}
			\hline
			\textbf{\begin{tabular}[c]{@{}c@{}}Количество \\ РПД\end{tabular}} & \textbf{Время без кэширования, мс} & \textbf{Время с кэшированием, мс} \\ \hline
			1 & 90213 & 6521 \\ \hline
			5 & 360041 & 24664 \\ \hline
			10 & 519212 & 40227  \\ \hline
			25 & 1881132 & 109710 \\ \hline
			100 & 7796774 & 754991 \\ \hline
		\end{tabular}%
	}
\end{table}

На рисунках \ref{plt:experiment1} - \ref{plt:experiment3} представлены графики зависимости количество запрашиваемых РПД от времени, при разных размерах кэша.

\begin{figure}[H]
	\centering
	\begin{tikzpicture}
	\begin{axis}[
	axis lines=left,
	xlabel=Количество запрашиваемых РПД,
	ylabel={Время, мс},
	legend pos=north west,
	ymajorgrids=true
	]
	\addplot table[x=RPD,y=ms,col sep=comma]{csv/cache_01.csv};
	\addplot table[x=RPD,y=ms,col sep=comma]{csv/storage_01.csv};
	\legend{С кэшированием, Без кэширования}
	\end{axis}
	\end{tikzpicture}
	\captionsetup{justification=centering}
	\caption{Зависимость времени от количества запрашиваемых РПД (размера кэша 100 элементов)}
	\label{plt:experiment1}
\end{figure}

\begin{figure}[H]
	\centering
	\begin{tikzpicture}
	\begin{axis}[
	axis lines=left,
	xlabel=Количество запрашиваемых РПД,
	ylabel={Время, мс},
	legend pos=north west,
	ymajorgrids=true
	]
	\addplot table[x=RPD,y=ms,col sep=comma]{csv/cache_02.csv};
	\addplot table[x=RPD,y=ms,col sep=comma]{csv/storage_02.csv};
	\legend{С кэшированием, Без кэширования}
	\end{axis}
	\end{tikzpicture}
	\captionsetup{justification=centering}
	\caption{Зависимость времени от количества запрашиваемых РПД (размера кэша 1000 элементов)}
	\label{plt:experiment2}
\end{figure}

\begin{figure}[H]
	\centering
	\begin{tikzpicture}
	\begin{axis}[
	axis lines=left,
	xlabel=Количество запрашиваемых РПД,
	ylabel={Время, мс},
	legend pos=north west,
	ymajorgrids=true
	]
	\addplot table[x=RPD,y=ms,col sep=comma]{csv/cache_03.csv};
	\addplot table[x=RPD,y=ms,col sep=comma]{csv/storage_03.csv};
	\legend{С кэшированием, Без кэширования}
	\end{axis}
	\end{tikzpicture}
	\captionsetup{justification=centering}
	\caption{Зависимость времени от количества запрашиваемых РПД (размера кэша 5000 элементов)}
	\label{plt:experiment3}
\end{figure}

\section*{Вывод}

В результате сравнения времени, необходимого для получения данных о рабочих программах дисциплин, кэширование данных показало неоднозначные результаты: 

\begin{itemize}
	\item приложение с кэшированием данных всегда работает быстрее;
	\item при запросах РПД превышающих максимальный размер кэша, приложение с кэшированием выигрывает по времени у обычного приложения без кэширования в среднем в 1.05 раза (вся эффективность кэширования практически нивелируется);
	\item если количество РПД в запросе не превышает максимальный размер кэша, выигрыш по времени в среднем составляет 12 раз (при условии, что выборочные данные находятся в кэше).
\end{itemize}

Эффективность кэширования данных в разработанном приложении полностью зависит от размера хранимых данных. Например, если есть возможность выделить кэш с максимальным размером хотя бы 30-50\% от максимально возможного размера хранимых данных, это обеспечит эффективность по времени максимум в 12 раз -- врядли кто-то будет запрашивать более 30\% данных расположенных в хранилище. 

Такой выигрыш по времени можно обеспечить только при условии что выборочные данные находятся в кэше, что, конечно, нереалистично. Можно сделать предположение, что хотя бы 20\% (для этого нужно выбрать подходящую политику вытеснения из кэша) выборочных данных находятся в кэше -- даже в таком случае доступ к данным будет ускорен в 2.4 раза.